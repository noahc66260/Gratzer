\documentclass[12pt]{amsart}
\usepackage{amssymb, latexsym, verbatim}
% \renewcommand{\baselinestretch}{1.5} % enable to get double spacing!

\theoremstyle{plain}
\newtheorem{name of environment}{Typeset Proclamation} %This is the format of a theorem like structure
\newtheorem{something}[name of environment]{Something}
\newtheorem*{no numbering}{No Numbering}
\newtheorem{proclamation}{Proclamation}[section]
\newtheorem{haha}{Haha}[section]
\newtheorem{bruhaha}[haha]{Bruhaha}

\newtheorem{plain}{Plain}

\theoremstyle{definition}
\newtheorem{definition}{Definition}

\theoremstyle{remark}
\newtheorem{remark}{Remark}

\begin{document}

	\title{Chapter 6: Text environments}
	\author[N.\, C. Ruderman]{Noah~C. Ruderman}

	\begin{abstract}
		We will review
		\begin{center}
		\begin{itemize}
			\item General rules for displayed text environments
			\item List environments
			\item Style and size environments
			\item Proclamations (theorem-like structures)
			\item Proof environments
			\item Tabular environments
			\item Tabbing environments (to be added later)
			\item Miscellaneous displayed text environments
		\end{itemize}
		\end{center}
	\end{abstract}

	\maketitle 
	% If the \maketitle command is missing, the title won't format 
	% and won't show up on the document

This is a numbered list made with the \texttt{enumerate} environment:
\begin{enumerate}
\item The first item \label{first}
\item The second item \label{second}
\item[] The third item
\end{enumerate}
Notice that there are no newline commands between the items. They are by default on separate lines. We see that \eqref{first} includes the text "The first item" and that we can use \textbackslash \texttt{eqref} for references that are not equations. We also see that the command \textbackslash \texttt{item[]} gives us an unnumbered item.
\vspace{12pt}

We can get a bulleted list with the \texttt{itemize} environment: 
\begin{itemize}
\item The first item
\item The second item
\item[] The third item
\end{itemize}
We see similar commands when we compare the \texttt{enumerate} and \texttt{itemize} environments. This includes the command \textbackslash \texttt{item[]} to get an unbulleted item. 

We can get a captioned list with the \texttt{description} environment:
\begin{description}
\item[1] This is our first item
\item[Two] This is our second item
\item[She sells sea shells] by the sea shore
\item[] This is our fourth item
\end{description}
We see that the optional argument for the command \textbackslash \texttt{item} is the caption of each item and we cannot omit it and expect favorable results.
\vspace{12pt}

Let's try nesting list environments. The limit is four.
\begin{enumerate}
\item This is first item of level 1
\item This is  second item of level 1
	\begin{enumerate}
	\item This is the first item of level 2
		\begin{enumerate}
		\item This is the first item of level 3 \label{first item level 3}
		\item This is the second item of level 3
			\begin{enumerate}
			\item This is the first item of level 4 \label{first item level 4}
			\end{enumerate}
		\end{enumerate}
	\item This is the second item of level 2
	\item This is the third item of level 2
	\end{enumerate}
\end{enumerate}
We see that we can reference items such as the first item of level 4 which is item \ref{first item level 4}. 
\newpage
Let's try the same for bulleted lists:
\begin{itemize}
\item This is first item of level 1
\item This is  second item of level 1
	\begin{itemize}
	\item This is the first item of level 2
		\begin{itemize}
		\item This is the first item of level 3
		\item This is the second item of level 3
			\begin{itemize}
			\item This is the first item of level 4
			\end{itemize}
		\end{itemize}
	\item This is the second item of level 2
	\item This is the third item of level 2
	\end{itemize}
\end{itemize}
\vspace{12pt}

What about descriptions:
\begin{description}
\item[first] This is first item of level 1
\item[second] This is  second item of level 1
	\begin{description}
	\item[third] This is the first item of level 2
		\begin{description}
		\item[fourth] This is the first item of level 3
		\item[fifth] This is the second item of level 3
			\begin{description}
			\item[sixth] This is the first item of level 4
			\end{description}
		\end{description}
	\item[seventh] This is the second item of level 2
	\item[eighth] This is the third item of level 2
	\end{description}
\end{description}
We can see that nested \texttt{description} environments aren't too helpful.
\vspace{12pt}

There are several environments for aligning text.
\begin{flushright}
We can use the \texttt{flushright} environment to right align text.
\end{flushright}
\begin{flushleft}
We can use the \texttt{flushleft} environment to left align text.
\end{flushleft}
\begin{center}
Or we can use the \texttt{center} environment to center the text.
\end{center}
Take note of the vertical spacing between the \\
environments that does not exist otherwise. \par
If you don't want that vertical spacing between lines, you can use the command declarations: \\
\raggedright use the \textbackslash \texttt{raggedright} command to left-align.\\
\raggedleft use the \textbackslash \texttt{raggedleft} command to right-align.\\
\centering finally use the \textbackslash \texttt{centering} command to center-align. 
\vspace{12pt}

\begin{name of environment} % see how we invoke the environment with the designated name of the environment.
This is the theorem-like structure I created.
\end{name of environment}
\vspace{12pt}
\newpage

If the theorem-like environment starts with a list, you are supposed to include \textbackslash \texttt{hfill} before we declare the list. We can see why below. The first environment doesn't use the command \textbackslash \texttt{hfill} while the second one does. 

\begin{name of environment}
	\begin{enumerate}
	\item this is the first item
	\end{enumerate}
\end{name of environment}

\begin{name of environment}
	\hfill
	\begin{enumerate}
	\item this is the first item
	\end{enumerate}
\end{name of environment}

Notice that this text is centered because we never changed things back to how they were orginally.
\vspace{12pt}

\raggedright Now things should be better. \\
To type two sets of proclamations (theorem-like environments) with consecutive numbering, we must make the optional argument of the second declared theorem like structure the environment name of the first. 

\begin{something}[this is an optional argument]
See how this is numbered consecutively with the proclamation \textbf{Typeset Proclamation}?
\end{something}

\begin{proclamation}
Here we see that we can number theorem-like structures just like a section. We will see that we can number a theorem-like structure like any section command provided by the document class which includes section, subsection, chapter. 
\end{proclamation}
The general form of a \texttt{newtheorem} environment is \\ \textbackslash newtheorem\{\emph{envrname}\}[\emph{procCounter}]\{\emph{Name}\}[\emph{secCounter}] in which the optional arguments are mutually exclusive, meaning that they cannot both exist at the same time.
\begin{haha}
haha!
\end{haha}

\begin{bruhaha}
bruhaha!
\end{bruhaha}
I seem to be unable to format the proclamations so that I can get numbering like 1.1, 1.2, 1.3, 2.1, 2.2, etc.
\vspace{12pt}

We see that we can get unnumbered theorems:
\begin{no numbering}
This theorem is unnumbered.
\end{no numbering}
We are also able to have different styles for our theorems:
\begin{plain}
This is the formatting for the \emph{plain} theorem-style. This is also the default if not specified.
\end{plain}
\begin{definition}
Another option is the \emph{definition} style. This is less emphatic than the \emph{plain} style.
\end{definition}
\begin{remark}
Our final option is the \emph{remark} style. Notice that it is the least emphatic.
\end{remark}
\begin{proclamation} \hfill
\begin{proof}
This a proof, and there is a q.e.d. symbol at the end. Notice how no theorem-like structure was declared for the proof environment. It is probably built in. Also note that this is an environment within an environment. 
\end{proof}
\end{proclamation}
Also, for a proof environment, we can substitute the word \emph{proof} for something else by adding an optional argument. Also, in the event that normal typesetting does not place the q.e.d. symbol correctly, we can force the printing of the q.e.d. symbol with the command \textbackslash \texttt{qedhere} command. 
\begin{proof}[PROOF]
Here is the equation of a circle:
\[
x^2 + y^2 = r^2 \qedhere
\]
\end{proof}
\vspace{12pt}

Something cool to know is that tables cannot be broken across pages like they can in microsoft word. Also, tables are treated as a single large symbol. We can also create a \texttt{table} environment in which the table is set off from surrounding text with vertical space and we can specify where we want our table to appear using float controls.
\begin{table}[h]
	\begin{center}
		\begin{tabular}{| l | c | c | c|} % r = right align, c = center align, l = left align
		\hline % I am unsure what the purpose of this part is
		{}	& Jonathan 	& Kerry	& Bob \\ \hline
		Age 	& 19 		& 26 		& 12  \\ \hline
		Eye color 	& brown 	& hazel 	& blue \\ \hline
		\end{tabular}
		\caption{This is our caption} \label{table1}
	\end{center}
\end{table}
\\We used the optional argument [\texttt{h}] to create the table where the environment was declared instead of at the top or bottom of the page. Note that we format the alignment of the text within columns. 
\\
We have the option of specifying the width of individual columns if we like with the \texttt{p}\{\emph{width}\} specifier: \\
\begin{center}
	\begin{tabular}{| l | p{2in} | c | c|} % notice the p{width} specifier in the second column
	\hline 
		& \centering Jonathan 	& Kerry	& Bob \\ \hline
	Age 	& \centering 19 		& 26 		& 12  \\ \hline
	Eye color 	& \centering brown 	& hazel 	& blue \\ \hline
	\end{tabular}
\end{center}
Notice how we must align the text by using commands on the individual elements of each column rather than when we begin the tabular environment.

We see that the midline characters denote that there will be a horizontal line at that point in the table. Omitting them omits the horizontal line like such:
\begin{table}[h]
	\begin{center}
		\begin{tabular}{l c c c} 
		\hline 
		{}	& Jonathan 	& Kerry	& Bob \\ \hline
		Age 	& 19 		& 26 		& 12  \\ \hline
		Eye color 	& brown 	& hazel 	& blue \\ \hline
		\end{tabular}
	\end{center}
\end{table}
Also, that \textbackslash \texttt{hline} command creates a horizontal line that we see in the table. Omiiting them removes the corresponding horizontal lines:
\begin{table}[h]
	\begin{center}
		\begin{tabular}{| l | c | c | c |}  
		{}	& Jonathan 	& Kerry	& Bob \\ 
		Age 	& 19 		& 26 		& 12  \\ 
		Eye color 	& brown 	& hazel 	& blue \\ 
		\end{tabular}
	\end{center}
\end{table}
\\We can mix and match accordingly:
\begin{table}[h]
	\begin{center}
		\begin{tabular}{| l  c | c | c }  
		{}	& Jonathan 	& Kerry	& Bob \\ \hline
		Age 	& 19 		& 26 		& 12  \\ 
		Eye color 	& brown 	& hazel 	& blue \\ 
		\end{tabular}
	\end{center}
\end{table}
\\ Two more features are the command \textbackslash \texttt{cline}\{\texttt{a-b}\} in which a horizontal line is drawn from column \texttt{a} to column \texttt{b} and the \textbackslash \texttt{multicolumn}\{\emph{number cols}\}\{\emph{alignment}\}\{\emph{text}\} command which merges row. Here's an example:
\begin{table}[h]
	\begin{center}
		\begin{tabular}{| l |  c | c | c | }  
		\hline
		\multicolumn{2}{| r |}{Jonathan} 	& Kerry	& Bob \\ \hline
		Age 	& 19 		& 26 		& 12  \\  \cline{2-4}
		Eye color 	&  brown 	& hazel 	& blue \\ \cline{3-4}
		\end{tabular}
	\end{center}
\end{table}
\\ Notice how the vertical lines are omitted on either side unless specified in the alignment argument of the \textbackslash \texttt{multicolumn} command. Below I am going to recreate the table from page 137 of Gratzer:
\begin{table}
	\begin{center}
		\begin{tabular}{| c c | c | r|}
		\hline
		Name 	& Month 	& Week 	&  Amount \\ \hline
		Peter 	& Jan. 		& 1 		& 1.00 \\ \cline{3-4}
			&		& 2		& 12.78 \\ \cline{3-4}
			&		& 3 		& 0.71 \\ \cline{3-4}
			& 		& 4 		& 15.00 \\ \cline{2-4}
			& \multicolumn{2}{| l}{Total} & 29.49 \\ \hline
		John	& Jan. 		& 1		& 12.01 \\ \cline{3-4}
			& 		& 2		& 3.10 \\ \cline{3-4}
			&		& 3 		& 10.10 \\ \cline{3-4}
			& 		& 4 		& 0.00 \\ \cline{2-4}
			& \multicolumn{2}{| l}{Total} 	& 25.21 \\ \hline
			\multicolumn{3}{| l }{ 
			\parbox[b]{1.5in}{	\strut Grand Total \\
						of the order}} & 54.70 \\ \hline
		\end{tabular}
		\caption{Table 6.3: Table with \textbackslash \texttt{multicolumn} and \textbackslash \texttt{cline}.}
	\end{center}
\end{table}
\newpage
Yes, you can use the \textbackslash \texttt{parbox} command to insert a single multiline entry. Also note the use of the \textbackslash \texttt{strut} command so that the text isn't squished. 
\vspace{12pt}

The vertical spacing of the table can be adjusted:
	\begin{center}
	\renewcommand{\arraystretch}{1.5}
		\begin{tabular}{| l | c | c | c|} 
		\hline 
		{}	& Jonathan 	& Kerry	& Bob \\ \hline
		Age 	& 19 		& 26 		& 12  \\ \hline
		Eye color 	& brown 	& hazel 	& blue \\ \hline
		\end{tabular}
	\end{center}
\vspace{12pt}

Here is a simple quote:
\begin{quotation}
It's not that I'm afraid to die. I just don't want to be there when it happens. \\
\begin{flushright}
\emph{Woody Allen}
\end{flushright}
\end{quotation}
Finally we arrive at the \texttt{verbatim} environment!. Here's what we can type:
\begin{verbatim}
The general form of a newtheorem environment is
\newtheorem{envrname}[procCounter]{Name}[secCounter] 
in which the optional arguments are mutually exclusive, 
meaning that they cannot both exist at the same time.
\end{verbatim}
Alternatively, we could use the \verb+\verb|xxx|+ command. The midline characters are arbitrary delimiters and the text \texttt{xxx} is arbitrary text with is printed verbatim. Try this: \\
The general form of a newtheorem environment is \verb|\newtheorem{envrname}[procCounter]{Name}[secCounter]| in which the optional arguments are mutually exclusive, meaning that they cannot both exist at the same time.
\vspace{12pt}

In the event that the delimiter is used in the text, just use a different delimiter. Anything will do, except an asterisk (*). Also, there is a special version of the \texttt{verbatim} environment and \verb+\verb+ command. Add an asterisk to show the spaces like such:
\verb*|today is Friday|

\end{document}
