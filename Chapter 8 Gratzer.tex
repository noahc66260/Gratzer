\documentclass{amsart}
\usepackage{amscd, verbatim}

\DeclareMathOperator{\boldsum}{\pmb{\sum}}
\DeclareMathOperator*{\boldsumlim}{\pmb{\sum}}

\begin{document}

	\title{Chapter 8: More Math}
	\author[N.\, C. Ruderman]{Noah~C. Ruderman}

	\begin{abstract}
		We will review
		\begin{center}
		\begin{itemize}
			\item Spacing of symbols
			\item Building new symbols
			\item Math alphabets and symbols
			\item Vertical spacing
			\item Tagging and grouping
			\item Miscellaneous
		\end{itemize}
		\end{center}
	\end{abstract}

	\maketitle 
	% If the \maketitle command is missing, the title won't format 
	% and won't show up on the document

	\noindent
	To insert a thin space, we type \verb+\,+. 
	Compare this
	\[
		A = \{ x \in X \mid x \beta \geq xy > (x + 1)^2 - \alpha \}
	\]
	with this
	\[
		A = \{\, x \in X \mid x \beta \geq xy > (x + 1)^2 - \alpha \, \}
	\]
	We see that there is more space between the brackets and what lies
	inside.
	Also something important to note is the difference between the 
	\verb+\mid+ command and the use of a \verb+|+, which is a midline 
	character.
	Using the \verb+\mid+ command inserts a \verb+|+ character which 
	acts as a binary operator, so spacing is automatic. 
	Using the \verb+|+ character by itself does not act as a binary
	operator, so there is no implicit spacing. 
	Compare the following: 
	\begin{align*}
		X &\mid x 	&& \text{ uses the } 
						\texttt{\textbackslash mid} \text{ command }\\
		X &| x			&& \text{ uses the } | \text{ character }
	\end{align*}
	\LaTeX{} implicitly creates spacing for binary operators, such as 
	the + and - operators.
	If you intend for a that can be binary or unary to be treated as
	a binary operator, place empty brackets \{ \} one one side of the 
	operator. 
	Compare $+a$ and ${}+a$. 

	There are many different spacing commands, and spacing commands 
	may either be positive or negative. 
	In order of increasing width, the positive spaces are: 
	\verb+\,+ or \verb+\thinspace+; \verb+\:+ or \verb+\medspace+; 
	\verb+\;+ or \verb+\thickspace+; or \verb+\ + which is an 
	interword space. 
	Also there is \verb+\quad+, which is 1 em long and 
	\verb+\qquad+, which is 2 em long. \\
	\begin{center}
		t\,h\,i\,n\,s\,p\,a\,c\,e \\
		m\:e\:d\:s\:p\:a\:c\:e \\
		t\;h\;i\;c\;k\;s\;p\;a\;c\;e \\
		i\ n\ t\ e\ r\ w\ o\ r\ d\quad s\ p\ a\ c\ e
	\end{center}
	To use negative spaces, generally the long command must be used.
	Here are the options: \verb+\!+ or \verb+\negthinspace+, 
	\verb+\negmedspace+, and \verb+\negthickspace+. 
	\begin{center}
		X\!x \\
		X\negmedspace{}x \\
		X\negthickspace{}x
	\end{center}
	Compare the following
	\begin{align*}
		\sqrt{5} \text{side} \quad &\text{to} \quad \sqrt{5} \, \text{side} \\
		\sin x / \log n \quad 		&\text{to} \quad \sin x / \! \log n \\
		f(1 / \sqrt{n}) \quad 		&\text{to} \quad f(1 / \sqrt{n}\,) \\
		f: A \to B \quad 				&\text{to} \quad f \colon A \to B
	\end{align*}
	The \verb+\phantom+ command is very useful but examples won't be
	developed here. 

	The \verb+\overset+ and \verb+\underset+ commands allow new symbols
	to be made from extant ones. 
	There are two arguments; the first is the variable to be displayed
	above or below the second argument.
	For underset, $\underset{\text{first argument}}{\text{second argument}}$ 
	and for overset, $\overset{\text{first argument}}{\text{second argument}}$.
	We can get creative with this command
	\[
		\overset{l}{+} 	\quad
		\overset{\alpha}{x} \quad
		\overset{\overset{\alpha}{x}}{x} \quad
		\underset{x \to \infty}{\lim x} \quad
		\underset{j \, \text{even}}{A_j}
	\] 
	Generally with a large operator, we can place subscripts or 
	superscripts
	\[
		\sum_{i = 0}^j i = \frac{j(j+1)}{2}
	\]
	However, in the event that we would like to get more creative with 
	where we may place symbols around a large operator, we may use
	the four corners by use of the \verb+\sideset+ command, which
	has the form \\
	\verb+\sideset{ _{ll}^{ul} }{ _{lr}^{ur} }{large_op}+
	and from this we get constructions such as 
	\[
		\sideset{ _{ll}^{ul} }{ _{lr}^{ur} }{\prod}_b^t
	\]
	If we want to use a symbol as a binary operator, we can tell
	\LaTeX{} of our intentions so it spaces accordingly. 
	This is done with the 
	\verb+\mathbin{operator}+ or \verb+\mathrel{operator}+ commands. 
	The first treats the symbol as a binary operator, the second
	treats the symbol as a binary relation. 
	\[
		A \alpha B \quad \text{compare to} \quad A \mathbin{\alpha} B
	\]
	
	When you type letters in a math formula you are using a 
	\emph{math alphabet}. 
	\begin{align*}
		\mathbf{Math\ bold:\ 2\ Greek\ gammas,\ \gamma\ and\ \Gamma} \\
		\mathit{Math\ italics:\ 2\ Greek\ gammas,\ \gamma\ and\ \Gamma} \\
		\mathsf{Math\ sans\ serif:\ 2\ Greek\ gammas,\ \gamma\ and\ \Gamma} \\
		\mathrm{Math\ roman:\ 2\ Greek\ gammas,\ \gamma\ and\ \Gamma} \\
		\mathtt{Math\ typewriter:\ 2\ Greek\ gammas,\ \gamma\ and\ \Gamma} \\
		\mathnormal{Math\ normal:\ 2\ Greek\ gammas,\ \gamma\ and\ \Gamma}
	\end{align*}
	The respective commands to produce the text above are 
	\verb+\mathbf+\{\emph{text}\}, 
	\verb+\mathit+\{\emph{text}\}, 
	\verb+\mathsf+\{\emph{text}\}, 
	\verb+\mathrm+\{\emph{text}\}, 
	\verb+\mathtt+\{\emph{text}\}, 
	\verb+\mathnormal+\{\emph{text}\} (the default). 
	Note that the default setting is math italics. 
	
	Note: symbols and math alphabets are disjoint sets. 
	There are no italicized, slanted, or sans serif versions. 
	However, if you want to make a bold symbol, you can use the
	\verb+\boldsymbol+\{\emph{symbol}\} command. 
	Compare $\boldsymbol{\alpha}\ \text{to}\ \alpha$. 
	There are four alphabets of symbols in \LaTeX: 
	\begin{center}
		\begin{tabular}{ccc} %alphabet, command, examples
			Alphabet 	
				& Command 	
				& Examples \\ \hline \hline
			Greek 		
				& symbol specific 	
				& $\alpha$, $\beta$, $\gamma$ \\ \hline
			Calligraphic 
				& \verb+\mathcal{}+ 	
				& $\mathcal{A,\ B,\ C}$ \\ \hline
			Euler Fraktur 	
				& \verb+\mathfrak{}+ 
				& $\mathfrak{A,\ B,\ C}$ \\ \hline
			Blackboard bold 
				& \verb+\mathbb{}+ 	
				& $\mathbb{A,\ B,\ C}$
		\end{tabular}
	\end{center}
	You can, of course, combine commands relating to symbols.
	If you want to type $\boldsymbol{\mathcal{AMS}}$, you can't do so using
	the \verb+\mathbf{}+ command since the font is a symbol alphabet, not a
	math alphabet.
	But you can combine the \verb+\boldsymbol+\{\emph{symbols}\} with
	the Calligraphic alphabet.
	Try the command \verb+\boldsymbol{\mathcal{AMS}}+.
	Note: the command \verb+\mathcal{\boldsymbol{AMS}}+ doesn't work. 
	Can you guess why?

	Some symbols do not have a bold version, such as the $\sum$ symbol. 
	To get a bold version, use the \verb+\pmb+\{\emph{symbol}\} command.
	From this we get $\pmb{\sum}$.
	The \verb+\pmb{}+ command typesets three very closely spaced symbols
	to get a rudimentary bold version. 
	However, this does destroy \LaTeX's implicit spacing for the symbol, 
	which is no longer treated as a large operator. 
	Compare
	\[
		\sum_{i = 1}^n i^2 \quad
		\pmb{\sum}_{i = 1}^n i^2 \quad
		\boldsum_{i = 1}^n i^2 \quad
		\boldsumlim_{i=1}^n i^2
	\]
	The first variant is the regular one.
	The second variant use the \verb+\pmb{}+ command, which
	\LaTeX no longer treats as a large operator. 
	The third variant defines \verb+\pmb{\sum}}+ as a math operator
	with the command \\
	\verb+\DeclareMathOperator{\boldsum}{\pmb{\sum}}+
	in the header. 
	The fourth variant does the same as the third but uses the
	\verb+*+ed version: \\
	\verb+\DeclareMathOperator*{\boldsumlim}{\pmb{\sum}}+

	Vertical spacing is also an issue when typesetting formulas. 
	We can control vertical spacing with the \verb+{\mathstrut ...}+ 
	command or the \verb+\vphantom{...}+ command.
	the \verb+{\mathstrut ...}+ command inserts an invisible 
	vertical space.
	The \verb+\phantom{...}+ command measures the height of its argument
	and places a math strut of that height.
	Generally the second option is more versatile than the former. 
	Compare $\sqrt{a} + \sqrt{b}$ to 
	$\sqrt{\mathstrut a} + \sqrt{\mathstrut b}$ (\verb+\mathstrut+)
	or $\sqrt{\vphantom{b} a} + \sqrt{b}$ (\verb+\vphantom{}+). 

	We can use the \verb+\smash{...}+ command to tell \LaTeX{} 
	to pretend that $\smash{\frac{1}{\int_{-\infty}^{\infty} e^{x^2} dx}}$
	or an argument like it doesn't protrude above or below the 
	current line.

	We can attach tags to equations with the \verb+\tag+\{\emph{argument}\}
	command. 
	\begin{equation}
		\int_{-\infty}^{\infty} e^{x^2} dx \tag{my customized tag}
	\end{equation}
	Since the customized tag explicitly inserts a tag, it does not 
	matter if we use the \verb+\[ \]+, \verb+equation*+, or \verb+equation+
	environments. 
	Also, tags are absolute, so they will not change if the equation is 
	moved, unless the argument is a reference to a label.
	If we use the variant \verb+\tag*+\{\emph{argument}\}, the parentheses
	are omitted.
	\begin{equation}
		\int_{-\infty}^{\infty} e^{x^2} dx \tag*{my customized tag}
	\end{equation}
	
	To be able to reference equations, we use the 
	\verb+\label+\{\emph{argument}\} command. 
	We can have both a tag and a label
	\begin{equation} \label{q52}
		\int_{-\infty}^{\infty} ze^{z^2} dz \tag*{$q_{52}$}
	\end{equation}
	and we can now reference the equation with the 
	\verb+\ref+\{\emph{label}\} to get \ref{q52} or alternatively
	we may use \verb+\eqref+\{\emph{label}\} to get \eqref{q52}. 
	Since the tag we used was static, the reference will always yield 
	the same value, but if equation were labeled by some other means
	then the reference would yield the current label of the equation. 

	The \verb+subequations+ environment allows the implicit numbering
	of the equations to follow the pattern of 1a, 1b, ...
	\begin{subequations}
		\begin{equation}
			1 + 1 = 2
		\end{equation}
		\begin{equation}
			2 + 2 = 4
		\end{equation}
	\end{subequations}

	If you plan to do some creative things with fraction formatting,
	use the \\
	\verb+\genfrac+\{\emph{left-delim}\}
		\negthickspace\{\emph{right-delim}\}
		\negthickspace\{\emph{thickness}\}
		\negthickspace\{\emph{mathstyle}\}
		\negthickspace\{\emph{numerator}\}
		\negthickspace\{\emph{denominator}\}
	command. 
	The \emph{thickness} argument is the thickness of the fraction line, 
	which is in the form $x$pt (the default is 0.4 pt). 
	For the \emph{mathstyle} argument, choose 0 for \verb+\displaystyle+, 
	1 for \verb+\textstyle+, 2 for \verb+\scriptstyle+, or 3 for
	\verb+\scriptscriptstyle+ (the default is 0). 
	All argument must be specified for this command, but the empty
	argument gives the default value. 

	The \verb+\boxed{...}+ command creates a box around math formula.
	\begin{equation}
		\boxed{\int_{-\infty}^{\infty} e^{x^2} dx}
	\end{equation}

\end{document} 
