\documentclass{amsart}
\usepackage{amssymb, latexsym, verbatim}
\usepackage{ch15CommandFile}
\usepackage{layout}

\renewcommand{\thepage}{(\arabic{page} or \Roman{page})}
\newcommand{\tbs}{\textbackslash}
\newcommand{\ttt}[1]{\texttt{#1}}
\newcommand{\pqquad}{\phantom{\qquad}}
\newcommand{\emttt}[1]{\texttt{\emph{#1}}}

\begin{document}

	\title{Chapter 15: Customizing \LaTeX}
	\author[N.\, C. Ruderman]{Noah~C. Ruderman}

	\begin{abstract}
		We will review
		\begin{center}
		\begin{itemize}
			\item User-defined commands
			\item User-defined environments
			\item A custom command file
			\item Numbering and measuring
			\item Custom lists
		\end{itemize}
		\end{center}
	\end{abstract}

	\maketitle 
	% If the \maketitle command is missing, the title won't format 
	% and won't show up on the document

	User-defined commands, also called macros, can make inserting
	text or equations you use often easier.
	For example, if you use the equation 
	\begin{equation}\label{E:01}
		\int_0^{\infty} P(X = x) dx
	\end{equation}
	often, you may redefine it as a command. 
	To create a user defined command, we use the command
	\verb+\newcommand+\{\emph{command 1}\}\{\emph{command 2}\} in the
	header. 
	\emph{command 2} is the command we wish to make the equivalent
	of typing \emph{command 1}.
	So if we wanted to typeset equation \eqref{E:01} with the shorter 
	command \verb+\Px+, we would insert the following into the 
	header:
	\verb+\newcommand{\Px}{\int_0^{\infty} P(X = x) dx}+
	As proof we use the user-defined command below
	\begin{equation}
		\Px
	\end{equation}
	There are a few things to keep in mind:
	\begin{enumerate}
		\item
			The first argument must not be a command that is already in use
		\item
			The second argument must be spelled correctly, but \LaTeX{} will
			only give you an error when you try to use it if this is the case.
		\item
			Macros in \LaTeX{} work the same way as in C, so command 
			declarations will need an extra pair of braces. 
	\end{enumerate}

	In the above example when we defined the command \verb+\Px+, we 
	see that it is a math formula so we can only use it in a math
	environment. 
	Alternatively, we could have defined \verb+\Px+ to be
	\verb+$\int_0^{\infty} P(X = x) dx$+ so that it could be used
	in a text environment, but then this would prohibit use 
	in a math environment without \verb+$+ delimiters. 
	There is a solution around this.
	We can use the \verb+\ensuremath{...}+ command so that the 
	\verb+\Px+ command works in both environments. 
	We would change the original user defined command to the following:
	\verb+\newcommand{\Px}{\ensuremath{\int_0^{\infty} P(X = x) dx}}+.
	Now \verb+$\Px$+ and \verb+\Px+ typeset as \Px. 

	We can also make user-defined commands that take arguments.
	For example, if we wanted to define a 
	\verb+\pwr+\{\emph{base}\}\{\emph{exponent}\}
	command that prints $\text{base}^{\text{exponent}}$, we would
	put the following in the header: \\
	\verb+\newcommand{\pwr}[2]{\ensuremath{{\text{#1}}^{\text{#2}}}}+.
	We use it in the following: 
	\pwr{(argument #1)}{(argument #2)}.
	To define a command with arguments, we insert a number enclosed
	in brackets after the first set of braces.
	This number may take any value from 1-9 and represents the number
	of arguments that the command takes.
	In the second pair of braces, we type #1, #2, ... #n to insert
	the value of the \pwr{1}{st}, \pwr{2}{nd}, or \pwr{n}{th} argument.  

	We can make the first command optional and provide a default
	value for the optional argument. 	
	Suppose we would like a command that accepts two arguments (let's
	assume they're $a$ and $b$) and typesets $a_1 + a_2 + \cdots + a_b$. 
	We could put the following in the header: \\
	\verb!\newcommand{\seq}[2][n]!
		\!\verb!{\ensuremath{#2_1 + #2_2 + \cdots + #2_{#1}}}! \\
	This says that the \verb+\seq+ command takes two arguments, that the
	first argument is optional, and that the default value for the 
	first arguement is $n$.
	Now that the first argument is optional, we enclose it in brackets
	\verb+[ ]+ rather than braces \{ \}. 
	We omit the brackets, rather than providing them with no value to 
	tell \LaTeX\ that we would like to use the default value.
	\begin{align*}
		\text{\textbackslash}\texttt{seq\{x\}} &\longrightarrow \seq{x} \\
		\text{\textbackslash}\texttt{seq[]\{x\}} &\longrightarrow \seq[]{x} \\
		\text{\textbackslash}\texttt{seq[z]\{x\}} &\longrightarrow 
			\seq[z]{x} \\
	\end{align*}
	
	We may also redefine commands, although it would be used sparingly. 
	For example, the QED symbol is typeset as \verb+\qedsymbol+ \qedsymbol.
	If we would like a solid black square $\blacksquare$ rather than
	the hollow square, we may add the following to the header:
	\verb+\renewcommand{\qedsymbol}{\ensuremath{\blacksquare}}+.
	Now  \verb+\qedsymbol+ appears as \qedsymbol. 

	This is somewhat unrelated, but the command \verb+\linewidth+ is the 
	horizontal spacing of a given line in the current format. 

	We can also create user defined environments.
	Here is a user-defined pseudo environment you've been using:
	\begin{verbatim}
		\phantom{\qquad}\textbf{a.} 
		\begin{minipage}[t]{11 cm}
		   Your text here.
		\end{minipage}
		\\[1ex]
	\end{verbatim}
	This pseudo-environment is the following:\\
	\phantom{\qquad}\textbf{a.} 
	\begin{minipage}[t]{11 cm}
		Your text here.
	\end{minipage}
	\\[1ex]
	Here's how you would redefine the environment to do such a thing:
	\begin{verbatim}
		\newlength{\cgap}
		\settowidth{\cgap}{\qquad \textbf{x. }}

		\newlength{\cwidth}
		\setlength{\cwidth}{\textwidth}
		\addtolength{\cwidth}{-\cgap}

		\newenvironment{cpart}[2][\cwidth]
		   {\\ \phantom{\qquad}\textbf{#2. }\begin{minipage}[t]{#1}}
		   {\end{minipage} \\}
	\end{verbatim}

	First we define a custom width called \verb+\cwidth+ for formatting
	purposes.
	Then we use the \verb+\newenvironment+ command.
	There are three mandatory arguments and two optional arguments.
	The first mandatory argument is the name of the environment, so that
	we may invoke with with \verb+\begin+\{\emph{environment name}\}. 
	Then there are two optional arguments;
	The first optional argument specifies the number of arguments
	that the environment takes, so we may now invoke it with 
	\verb+\begin+\{\emph{environment name}\}
		\!\!\{\emph{argument 1}\}\{\emph{argument 2}\}\dots
	The second optional argument implies that the first optional 
	argument is used and makes \emph{argument 1} optional. 
	The value of the second optional argument is the default value.
	As always, we place this all in the header.
	Now our environment invocation takes the form: \\
	\verb+\begin+\{\emph{environment name}\}
		\!\![\emph{argument 1}]\{\emph{argument 2}\}\dots\\ or
	\verb+\begin+\{\emph{environment name}\}
		\!\!\{\emph{argument 2}\}\dots (\emph{argument 1} takes default value)\\
	The second mandatory argument is what is inserted when we invoke
	our new environment with the \verb+\begin+\{\emph{environment name}\}\dots
	command and the third mandatory argument is what gets inserted when we
	end our environment with the \\ 
	\verb+\end+\{\emph{environment name}\} command. 
	
	This is what the environment looks like:
	\begin{cpart}{a}
		This is part (a), and as I type we will see if the formatting is correct
		by comparing where the line breaks off in this environment to outside
		of the environment. 
	\end{cpart}
	\begin{cpart}{b}
		This is part (b), and we can see the implicit spacing between the
		two environments. Here is an equation that is part of the environment:
		\[
			\Px
		\]
	\end{cpart}
	
	Instead of putting all of the user-defined commands in the header,
	you may choose to put them all in a custom command file.
	The program extension in the book for the command file is 
	\verb+.sty+, and I am unsure exactly why that is so. 
	You place everything in the command file as normal, but you
	must place the \verb+\endinput+ command at the end of the file.
	To include the file in the document, you use the 
	\verb+\usepackage+\{\dots\} command. 
	All of the commands used in this file, for example, 
	were moved into a file named \verb+ch15CommandFile.sty+ and 
	imported with the \verb+\usepackage{ch15CommandFile}+ command in the
	header.

	Counters are automatically handled by \LaTeX\ but sometimes
	custom formatting by the user is necessary.
	Examples of counters are the numbering of a theorem,
	section, equation, and page numbers.
	We can set extant counters with the
	\verb+\setcounter+\{\emph{counter}\}\{\emph{value}\} command. 
	We can also get the value of a counter by the 
	\texttt{\textbackslash the\emph{counter}} command.
	We see that the value of \verb+\theequation+ is \theequation. 
	Now we type \verb+\setcounter{equation}{999}+%
	\setcounter{equation}{999}
	and the value of \verb+\theequation+ is \theequation. 
	Now let's see what number we get when we type an equation:
	\begin{equation}
		\Px
	\end{equation}
	And now the value of \verb+\theequation+ is \theequation. 

	\LaTeX\ has standard counters used frequently with the following names \\
	\hfill	
	\begin{minipage}[t]{3 cm}
		\begin{itemize}
			\item equation
			\item figure
			\item footnote
			\item mpfootnote
			\item page
			\item table
		\end{itemize}
	\end{minipage} 
	\hfill
	\begin{minipage}[t]{4 cm}
		\begin{itemize}
			\item part
			\item chapter
			\item section 
			\item subsection
			\item subsubsection
			\item paragraph
		\end{itemize}
	\end{minipage} 
	\hfill
	\begin{minipage}[t]{4 cm}
		\begin{itemize}
			\item subparagraph
			\item enumi
			\item enumii
			\item enumiii
			\item enumiv
		\end{itemize}
	\end{minipage} \hfill
	
	We can choose to define our own counters with the 
	\verb+\newcounter+\{\emph{mycounter}\} command, to be placed in the
	header file.

	We can change the counter's appearance if we wish with the\\
	\verb+\renewcommand+\{\texttt{\textbackslash the\emph{counter}}\}%
		\{\emph{new\_style}\} command. 
	The following styles are available: \\
	\begin{center}
		\begin{tabular}{l l l}
			\hline
			Style 	
				& 	Command 	
				& Sample
 				\\ \hline
			Arabic 	
				& \texttt{\textbackslash arabic\{\emph{counter}\}}
				& 1, 2, \dots 
				\\
			Lowercase Roman
				& \texttt{\textbackslash roman\{\emph{counter}\}}
				& i, ii, \dots
				\\
			Uppercase Roman
				& \texttt{\textbackslash Roman\{\emph{counter}\}}
				& I, II, \dots
				\\
			Lowercase Letters
				& \texttt{\textbackslash alph\{\emph{counter}\}}
				& a, b, \dots, z
				\\
			Uppercase Letters
				& \texttt{\textbackslash Alph\{\emph{counter}\}}
				& A, B, \dots, Z
				\\ \hline
		\end{tabular}
	\end{center}
	We can get creative in how we redefine counters. 
	For example, the page numbers were redefined with the command \\
	\verb+\renewcommand{\thepage}{(\arabic{page} or \Roman{page})}+
	The following commands might be of use:
	\texttt{\textbackslash stepcounter\{\emph{counter}\}} increments
	\texttt{\emph{counter}} and sets all the counters that were 
	defined with the optional argument counter to 0.
	\texttt{\textbackslash addtocounter\{\emph{counter}\}\{\emph{n}\}}
	adds the value \texttt{\emph{n}} to \texttt{\emph{counter}}.
	
	\LaTeX\ has five absolute length units:
	\begin{itemize}
		\item \texttt{cm} centimeter
		\item \texttt{in} inch
		\item \texttt{pc} pica (1 pc = 12 pt)
		\item \texttt{pt} point (1 in = 72.27 pt)
		\item \texttt{mm} millimeter
	\end{itemize}
	and two relative ones:
	\begin{itemize}
		\item \texttt{em}, approximately the width of the letter M 
			in the current font.
		\item \texttt{ex}, approximately the width of the letter x
			in the current font.
	\end{itemize}
	See chapter 10 for common length commands. 

	We can define a new length command in the header like so:
	\verb+\newlength{\mylength}+.
	Now \verb+\mylength+ is interpreted as a length by \LaTeX. 

	To set the length of a length command, we type
	\verb+\setlength{\mylength}+\texttt{\emph{\{length\}}}.
	Some things to note: 
	\begin{enumerate}
		\item The first argument must be a length 
			command, which is to say that it must include the backslash 
			\textbackslash. 
		\item The second argument have units or be a length command. 
	\end{enumerate}

	Another way we can set the length of a command is to use the
	\verb+\settolength+\texttt{\{\emph{length command}\}\{\emph{argument}\}}.
	The last argument can be text or a math formula which is to be measured
	by \LaTeX\ whose length is assigned to the length command. 
	
	We can adjust the length of a length command with the 
	\verb+\addtolength+\texttt{\{\emph{length command}\}\{\emph{length}\}}
	command. 
	One thing to note here is that the last argument can take on negative
	values as well as positive ones. 
	Not including a sign in front of the last command means that \LaTeX\
	assumes you are implying the use of a positive value. 
	Also related to this command are the commands
	\verb+\settoheight+ and \verb+\settodepth+ which take the same
	arguments. 
	Note: "depth" is the length from the bottom of the line to the
	bottom of the box, which matters for letters like g and q. 

	Let's take a look at our previous use of length commands:
	\begin{verbatim}
		\newlength{\cgap}
		\settowidth{\cgap}{\qquad \textbf{x. }}

		\newlength{\cwidth}
		\setlength{\cwidth}{\textwidth}
		\addtolength{\cwidth}{-\cgap}

		\newenvironment{cpart}[2][\cwidth]
		   {\\ \phantom{\qquad}\textbf{#2. }\begin{minipage}[t]{#1}}
		   {\end{minipage} \\}
	\end{verbatim}
	We see that the judicious use of length commands can solve 
	problems that I used to find baffling. 

	It is possible to make custom lists, which addresses yet another
	problem I was having not very long ago. 
	First, we need to discuss the relevant length commands
	for a list environment.
	\begin{description}
		\item[\textbackslash \texttt{topsep}] 
			this is most of the vertical space between the first item
			and the preceding text and also most of the space between
			the last item and the following text. 
			The space aforementioned is also augmented by 
			\verb+\parskip+, the extra vertical space inserted between
			paragraphs. 
		\item[\textbackslash \texttt{parsep}] 
			this is the space between paragraphs of the same item.
		\item[\textbackslash \texttt{itemsep}]
			this is the space between items. Like \verb+\topsep+, the
			actual gap is the sum of \verb+\itemsep+ and \verb+\parsep+.
		\item[\textbackslash \texttt{leftmargin}]
			this is the distance between the edge of the item box and 
			the left margin of the page. I believe the item box
			is the text that follows \verb+\item+. 
		\item[\textbackslash \texttt{rightmargin}]
			analog of \verb+\leftmargin+ entry for right margin. 
		\item[\textbackslash \texttt{labelwidth}]
			the width of the box that the optional argument for
			the \verb+\item+ command is typeset in. 
		\item[\textbackslash \texttt{itemindent}]
			the amount of space that the box of length \verb+\labelwidth+
			is indented from the left margin. 	
		\item[\textbackslash \texttt{labelsep}]
			the space separation between the box that contains
			the label name and the box containing the text that follows. 
		\item[\textbackslash \texttt{listparindent}]
			the length of the indentation of the second and subsequent
			paragraphs of an item. 
	\end{description}
	
	Custom lists are created using the \verb+list+ environment, which
	is invoked as follows: 
	\ttt{\tbs begin\{list\}\{\emph{default\_label}\}\{\emph{declarations}\}} \\
	\pqquad \ttt{\tbs item \emph{item1}} \\
	\pqquad \ttt{\tbs item \emph{item2}} \\
	\pqquad \dots \\
	\ttt{\tbs end\{list\}}

	\emttt{default\_label} is the default label used for any items that
	do not specify their own label.
	\emttt{declarations} is the vertical and horizontal length commands 
	and any other required parameters for the list.

	Here's an example of a customized list (with counters):
	\newcounter{spacerule}
	\begin{list}{\textbf{Rule \arabic{spacerule}:}}
					{\setlength{\leftmargin}{3 cm}
					 \setlength{\rightmargin}{3 cm}
					 \setlength{\labelsep}{1 cm}
					 \usecounter{spacerule}}
		\item	
			We may use counters in our lists and we must specify that
			the counter will be used in the \emttt{default\_label} argument.
		\item
			If we use a counter and wish for it to be incremented
			automatically, we use the \\
			\verb+\usecounter{+\emttt{counter}\ttt{\}} command. 
	\end{list}
	Strangely, I am unable to get rid of the initial indentation of the
	first paragraph of an item. 
	If we happen to use a custom \ttt{list} environment often,
	we can define it as a new environment. 
	We can also apply labels to the items, and what will be recorded
	is the value of the counter for that item.

\end{document}
