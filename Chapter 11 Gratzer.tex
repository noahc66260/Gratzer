\documentclass[12pt, intlimits]{amsart} % The options are 9,10,11,12 pt and 12pt is useful for proofreading. intlimits allows the limits for an integral to be placed above and below it as opposed to the side. 
\usepackage{amssymb,latexsym} % If you add this, you can not worry about many things. Yeah, that was vague, but it was the author's recommendation to start off each document with \documentclass{amsart} and \usepackage{amssymb, latexsym} as a habit. 

\begin{document}
\title[This is the short Title]{The Argument here is the Long Version of the Title to be used in the Document}
\date{Place your date here, or use \texttt{\textbackslash today}} 
	% This shows up on the bottom of the page
\dedicatory{This is the optional dedicatory comment}
\author[N.\, C. Ruderman]{Noah~C. Ruderman}
\address{Place your address here}
\email[Author's Name]{Said author's email} % For multiple authors you may add the author's name as an optional argument
\email{You can add the email of second authors as well}
\urladdr{Place your website here} %again, an optional argument exists for the name of the author
\thanks{What is typeset here is under the "thanks" section}

\begin{abstract}
This is the abstract for the document and contains much relevant information.
\end{abstract}



\maketitle % If the \maketitle command is missing, the title won't format and won't show up on the document

Chapter 11: The AMS article document class

\bigskip

I don't understand why but the title and dedicatory commands are not working like they should. It actually turns out that I forgot to include the \emph{maketitle} command at the end of the top matter. This changed everything. 

\[
\int^{\infty}_{-\infty} f(x) \, dx
\]
% we see that the subscript and superscripts for the integral are located to the sides rather than the top and bottom of it. 

\end{document}
